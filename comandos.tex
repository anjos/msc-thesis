% $Id: comandos.tex,v 1.1 2000/11/03 20:47:00 andre Exp $

% Este fonte LaTeX deve ser inclu�do com um \input{} em outro fonte LaTeX.
% Ele consiste de novos comandos que facilitam certas atividades quando se
% est� escrevendo em LaTeX.

%% Estes comandos aqui definem novas cores. Para us�-las, voc� tem que
%% incluir no seu fonte, o pacote 'color'
\definecolor{azul}{rgb}{0.2,0.2,0.8}
\definecolor{verde}{rgb}{0.2,0.7,0.2}
\definecolor{vermelho}{rgb}{0.7,0.2,0.2}

%% Este comando inclui uma imagem, no formato EPS. Use o pacote 'graphicx'.
\newcommand{\epsimage}[3]{%
\begin{center}%
\includegraphics[type=eps,ext=.eps,scale=#1,bb=#2]{figures/#3}%
\end{center}}

%% Mudan�a de linguagem com o Babel. O pacote 'babel' deve ser usado.
\newcommand{\eng}[1]{\foreignlanguage{english}{\emph{#1}}}
\newcommand{\fr}[1]{\foreignlanguage{french}{\emph{#1}}}

%% Alguns atalhos
\newcommand{\eiro}{$^{\underline{o}}$} % Desenha o 'o.' do 1o.
\newcommand{\eira}{$^{\underline{a}}$} % Desenha o 'a.' de 1a.
\newcommand{\expo}[2]{$#1^{#2}\/$} % exponencia o 1o. arg com o 2o.
\newcommand{\raw}[1]{{\tt #1}} % Vai para modo truetype (raw text)

%% Para o novo ambiente algoritmo
\floatname{algorithm}{Implementa��o}
\renewcommand{\listalgorithmname}{Lista de Implementa��es}

%% Para classes
\newcommand{\classe}[1]{\foreignlanguage{english}{\texttt{#1}}}
\newcommand{\kword}[1]{\underline{\textsc{\color{azul}{#1}}}}

%% Define um ambiente para a ficha bibliogr�fica
\newenvironment{ficha}{%
\newlength{\largura}% calcula a largura de 40 colunas verbatim
\settowidth{\largura}{\ttfamily aaaaaaaaaaaaaaaaaaaaaaaaaaaaaaaaaaa}%
\begin{center}%
\begin{minipage}{\largura}%
\newlength{\saveparid}% cria um reposit�rio
\setlength{\saveparid}{\parindent}% guarda o valor default
\setlength{\parindent}{0.6cm}}%
{%
\setlength{\parindent}{\saveparid}% restaura o valor default
\end{minipage}%
\end{center}}

%% Define um ambiente para as dedicat�rias e agradecimentos
\newenvironment{dedicate}[2]{%
\begin{flushright}%
\begin{minipage}{#1}%
\begin{center}
#2
\end{center}
\setlength{\saveparid}{\parindent}% guarda o valor default
\setlength{\parindent}{0.6cm}}%
{%
\setlength{\parindent}{\saveparid}% restaura o valor default
\end{minipage}%
\end{flushright}}
	
%% Define um ambiente para digitar c�digo
\newenvironment{codigo}%
{\begin{alltt}\renewcommand{\baselinestretch}{1}\small}%
{\normalsize\renewcommand{\baselinestretch}{1.5}\end{alltt}}%

%% Atalhos
\newcommand{\etem}{E$^{\text{e.m.}}_{T_{3\times7}}$}
\newcommand{\ethad}{E$^{\text{HAD}}_{T_{0,2\times0,2}}$}
\newcommand{\rshape}{R$^{\text{shape}}_{\eta}$}
\newcommand{\rstrip}{R$^{\text{strip}}_{e.m._1}$}
\newcommand{\ep}{$\eta\times\phi$}





